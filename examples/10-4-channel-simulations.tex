# Multi-dimensional simulations

Models in Myokit use:
\begin{linenomath}
\begin{equation}
\frac{\partial v}{\partial t}=-\frac{1}{C_{m}}(i_{\mathit{ion}}+i_{\mathit{stim}}+i_{\mathit{diff}})\label{eq:dv/dt}
\end{equation}
\end{linenomath}
where $i_\mathit{diff}$ is a diffusion current defined as
\begin{linenomath}
\begin{equation}
i_{\mathit{diff},i}=\sum_{j\in N_{i}}g_{ij}\left(V_{i}-V_{j}\right)\label{eq:idiff}
\end{equation}
\end{linenomath}
where $N_{i}$ is the set of neighbours of cell $i$ and $g_{ij}$
is the conductance between cell $i$ and its neighbour $j$. In the
current simulation classes, only two conductances ($g_{x}$ and $g_{y}$)
can be specified and all cells are assumed to be in a regular grid.
This means all central cells have two x-neighbours and two y-neighbours,
while cells at the boundaries have 2 or 3 neighbouring cells. In other
words:
\begin{linenomath}
\begin{equation}
i_{\mathit{diff},i}=g_{x}\sum_{x\in N_{x}}\left(V_{i}-V_{x}\right)+g_{y}\sum_{y\in N_{y}}\left(V_{i}-V_{y}\right)
\end{equation}
\end{linenomath}
where $N_{x}$ and $N_{y}$ are the set of neighbours in the $x$
and $y$ direction respectively.


## Mono/Bi-domain equations

The bidomain model starts from the assumption of ohmic conductance
currents as in \ref{eq:idiff}. It then adds a few components representing
the extracellular conductance and generalises the result to a continuous
PDE. The mono-domain is derived from the bi-domain model by dropping
the terms explicitly modelling the extracellular conductance. Many
modelers then assume only conduction in the principal directions (x,y
and z) and add the boundary condition that no current enters or leaves
the system at the sides. Then, using a finite-difference approximation
of the second order derivatives arising from the bidomain equations,
we arrive right back at the original equation specifying ohmic conductance
between neighbouring cells.

Apart from the historical and mathematical confusion this propagates,
it has the disadvantage of introducing extra parameters with a less
clear meaning than $g_{x}$ and $g_{y}$, each of which needs to be
estimated. However, in order to compare results in Myokit with mono-domain
simulations, it may be necessary to translate beteen the used parameter
sets. Their relationship is derived below.

According to the mono-domain model, each point in the heart can be
described as:
\begin{linenomath}
\begin{equation}
\frac{\lambda}{1+\lambda}\nabla\cdot\left(\Sigma_{i}\nabla v\right)=\chi\left(C_{m}\frac{\partial v}{\partial t}+i_{\mathit{ion}}+i_{\mathit{stim}}\right)\label{eq:monodom}
\end{equation}
\end{linenomath}
With:
\begin{itemize}
\item $\lambda$ The intra- to extracellular conductivity ratio
\item $\Sigma_{i}$ A tensor representing the intracellular conductivity
\item $v$ The local membrane potential
\item $\chi$ The surface area of the membrane per unit volume
\item $C_{m}$ Capacitance per unit area
\item $i_{ion}$ Ionic current per unit area
\end{itemize}
Assuming all currents are normalised to unit area, we can combine
this with \ref{eq:dv/dt} to find the mono-domain equation for $i_{diff}$:
\begin{linenomath}
\begin{eqnarray*}
\frac{\lambda}{1+\lambda}\nabla\cdot\left(\Sigma_{i}\nabla V\right) & = & \chi\left(C_{m}\frac{\partial V}{\partial t}+i_{\mathit{ion}}+i_{\mathit{stim}}\right)\\
\frac{\lambda}{\chi\left(1+\lambda\right)}\nabla\cdot\left(\Sigma_{i}\nabla V\right) & = & C_{m}\frac{\partial V}{\partial t}+i_{\mathit{ion}}+i_{\mathit{stim}}\\
\frac{\partial V}{\partial t} & = & \frac{-1}{C_{m}}\left(i_{\mathit{ion}}+i_{\mathit{stim}}-\frac{1}{\chi}\frac{\lambda}{\left(1+\lambda\right)}\nabla\cdot\left(\Sigma_{i}\nabla V\right)\right)\\
 & = & \frac{-1}{C_{m}}\left(i_{\mathit{ion}}+i_{\mathit{stim}}+i_{\mathit{diff}}\right)
\end{eqnarray*}
\end{linenomath}
so that
\begin{linenomath}
\begin{equation}
i_{\mathit{diff}}=-\frac{1}{\chi}\frac{\lambda}{\left(1+\lambda\right)}\nabla\cdot\left(\Sigma_{i}\nabla V\right)
\end{equation}
\end{linenomath}
For the 2D case, let
\begin{linenomath}
\begin{equation}
\Sigma_{i}=\sigma_{x}\hat{\mathbf{i}}+\sigma_{y}\hat{\mathbf{j}}
\end{equation}
\end{linenomath}
then
\begin{linenomath}
\begin{equation}
\Sigma_{i}\nabla V=\frac{\partial V}{\partial x}\sigma_{x}\hat{\mathbf{i}}+\frac{\partial V}{\partial y}\sigma_{y}\hat{\mathbf{j}}
\end{equation}
\end{linenomath}
and
\begin{linenomath}
\begin{equation}
\nabla\cdot\left(\Sigma_{i}\nabla V\right)=\frac{\partial^{2}V}{\partial x^{2}}\sigma_{x}+\frac{\partial^{2}V}{\partial y^{2}}\sigma_{y}
\end{equation}
\end{linenomath}
A finite difference approximation of $\frac{\partial^{2}V}{\partial x^{2}}$
is
\begin{linenomath}
\begin{eqnarray*}
\frac{\partial^{2}V}{\partial x^{2}} & = & \frac{V_{i-1}-2V_{i}+V_{i+1}}{\Delta x^{2}}\\
 & = & -\left(\frac{V_{i}-V_{i-1}}{\Delta x^{2}}+\frac{V_{i}-V_{i+1}}{\Delta x^{2}}\right)\\
 & = & \frac{-1}{\Delta x^{2}}\sum_{j\in N_{x}}\left(V_{i}-V_{x}\right)
\end{eqnarray*}
\end{linenomath}
where $N_{x}$ is the set of neighbours of cell $i$ in the $x$ direction.
Using this approximation, we find:
\begin{linenomath}
\begin{eqnarray*}
i_{\mathit{diff}} & = & \frac{\sigma_{x}}{\chi\Delta x^{2}}\frac{\lambda}{1+\lambda}\sum_{j\in N_{x}}\left(V_{i}-V_{x}\right)+\frac{\sigma_{y}}{\chi\Delta y^{2}}\frac{\lambda}{1+\lambda}\sum_{j\in N_{y}}\left(V_{i}-V_{y}\right)\\
 & = & g_{x}\sum_{x\in N_{x}}\left(V_{i}-V_{x}\right)+g_{y}\sum_{y\in N_{y}}\left(V_{i}-V_{y}\right)\\
 & = & \sum g_{ij}\left(V_{i}-V_{j}\right)
\end{eqnarray*}
\end{linenomath}

So to convert the four mono-domain parameters to a single conductance
we use:
\begin{linenomath}
\begin{eqnarray*}
g_{x} & = & \frac{\lambda\sigma_{x}}{\chi\left(1+\lambda\right)\Delta x^{2}}\\
g_{y} & = & \frac{\lambda\sigma_{y}}{\chi\left(1+\lambda\right)\Delta y^{2}}
\end{eqnarray*}
\end{linenomath}

